\documentclass{article}
\begin{document}
\section*{Math}
\subsection*{Discrete Convolution}
There are two ways of solving a discrete convolution problem which I will call "Table and Diagonals" and "Shifting and Co-efficients" which will be demonstrated below
\subsubsection*{"Table and Diagonals" Example}
Consider the table with 2 discrete signals $ [1,1,1,1,1] $ and $ [1,2,4] $.\\
\begin{tabular}{c|ccccc}
	&1&1&1&1&1\\
	\hline
	1&1&1&1&1&1\\
	2&2&2&2&2&2\\
	4&4&4&4&4&4
\end{tabular}\\
Drawing diagonal segments from the bottom left to the top right \textbf{then} taking the sum of the elements with in each segment gives the convolution.
\[ [1, 2+1, 4+2+1, 4+2+1, 4+2+1, 4+2, 4] \]
\[ [1, 3, 7, 7, 7, 6, 4] \]
\subsubsection*{"Shifting and Co-efficients" Method}
Consider the signals $ [2,0,-1] $ and $ [1, 2, 0, -3] $ for the method.\\
\begin{tabular}{cccccc}
	2&4&0&-6&&\\
	&0&0&0&0&\\
	&&-1&-2&0&3\\
	\hline
	2&4&-1&-8&0&3
\end{tabular}
\section*{General Electrical Concepts}
\subsection*{Nodal Analysis}
\subsection*{Mesh Analysis}\
\subsection*{Superposition}
\end{document}